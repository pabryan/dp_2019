% Created 2018-02-08 Thu 10:51
\documentclass[12pt]{article}
                           \usepackage[all]{pabmacros}
\usepackage[utf8]{inputenc}
\usepackage[T1]{fontenc}
\usepackage{fixltx2e}
\usepackage{graphicx}
\usepackage{longtable}
\usepackage{float}
\usepackage{wrapfig}
\usepackage[normalem]{ulem}
\usepackage{textcomp}
\usepackage{marvosym}
\usepackage[nointegrals]{wasysym}
\usepackage{latexsym}
\usepackage{amssymb}
\usepackage{amstext}
\usepackage{hyperref}
\tolerance=1000
\usepackage{amsmath}
\usepackage[bibstyle=alphabetic,citestyle=alphabetic,backend=bibtex]{biblatex}
\usepackage[small,compact]{titlesec}
\setlength{\textwidth}{19.7 cm}
\setlength{\hoffset}{-3.5 cm}
\setlength{\voffset}{-3.3 cm}
\addtolength{\textheight}{6.0 cm}

\setlength{\baselineskip}{12.1 pt}
\setlength{\parskip}{4.5 pt}
\setlength{\parindent}{0 pt}
\setlength{\abovedisplayskip}{0 pt}
\setlength{\belowdisplayskip}{0pt}
\setlength{\abovedisplayshortskip}{0 pt}
\setlength{\belowdisplayshortskip}{0 pt}

\bibliography{refs}
\AtEveryBibitem{\clearfield{doi}}
\AtEveryBibitem{\clearfield{url}}
\AtEveryBibitem{\clearfield{issn}}
\renewcommand*{\bibfont}{\footnotesize}

\author{Paul Bryan}
\date{}
\title{}
\hypersetup{
 pdfauthor={Paul Bryan},
 pdftitle={},
 pdfkeywords={},
 pdfsubject={},
 pdfcreator={Emacs 25.2.2 (Org mode 8.3.5)}, 
 pdflang={English}}
\begin{document}


\section*{D9 Ten career-best academic research outputs}
\label{sec:orgheadline1}

\begin{itemize}
\item * Ben Andrews, \textbf{Paul Bryan}. \emph{Curvature bounds by isoperimetric comparison for normalized Ricci flow on the two-sphere}. Calc. Var. Partial Differential Equations 39, 419--428, 2010.

\emph{Developed striking new techniques for obtaining direct curvature and isoperimetric control for the Ricci flow. Provides a very short and direct proof of the main convergence results for the Rici flow on the sphere by developing a viscosity comparison theory for the isoperimetric profile.}

\item * Ben Andrews, \textbf{Paul Bryan}. \emph{A comparison theorem for the isoperimetric profile under curve-shortening flow}. Comm. Anal. Geom. 19, 503--539, 2011.

\emph{Isoperimetric profile comparisons obtained for the curvature flow. These are similar to the Ricci flow, but includes boundary terms that make the analysis significantly more involved.}

\item * Ben Andrews, \textbf{Paul Bryan}. \emph{Curvature bound for curve shortening flow via distance comparison and a direct proof of Grayson's theorem}. J. Reine Angew. Math. 653, 179--187. 2011.

\emph{This is know by many experts as the "go to" proof of the main convergence results for the curve shortening flow. A very direct proof is obtained using distance comparison with sharp comparisons for a viscosity equation. Significant simplifications and clarity are obtained compared with previous proofs.}

\item * \textbf{Paul Bryan} \emph{Curvature bounds via an isoperimetric comparison for Ricci flow on surfaces} Ann. Scuola Norm. Sup. Pisa Cl. Sci September 2016, Volume 16, Issue 3

\emph{Extends the Ricci flow two sphere result to arbitrary closed surfaces. A number of innovations are introduced such as lifting the flow to the universal conver and dealing with the lack of compactness.}

\item \textbf{Paul Bryan}, Janelle Louie \emph{Classification of Convex Ancient Solutions to Curve Shortening Flow on the Sphere} J. Geom. Anal. April 2016, Volume 26, Issue 2, 858--872

\emph{Only classification result know for non-planar curve shortening flow. Initiated the Harnack inequality in non-Euclidean backgrounds and developed the Aleksandrov reflection technique.}

\item \textbf{Paul Bryan}, Mohammad Ivaki, Julian Scheuer \emph{Harnack inequalities for evolving hypersurfaces on the sphere} Comm. Anal Geom. (Accepted) 2016 arXiv: [math.DG] (available at \url{http://arxiv.org/abs/1512.03374})

\emph{First example of Harnack inequality for hypersurface flows in higher dimension non Euclidean Riemannian manifolds. Extends Hamilton's approach for the Mean Curvature Flow in Euclidean space to a broader class of flows and non-Euclidean backgrounds. This iniated the continuing study of Harnack inequalities in non Euclidean background spaces.}
\end{itemize}
\end{document}
