% Created 2018-02-09 Fri 11:56
% Intended LaTeX compiler: pdflatex
\documentclass[12pt]{article}
                           \usepackage[all]{pabmacros}
\usepackage[utf8]{inputenc}
\usepackage[T1]{fontenc}
\usepackage{fixltx2e}
\usepackage{graphicx}
\usepackage{longtable}
\usepackage{float}
\usepackage{wrapfig}
\usepackage[normalem]{ulem}
\usepackage{textcomp}
\usepackage{marvosym}
\usepackage[nointegrals]{wasysym}
\usepackage{latexsym}
\usepackage{amssymb}
\usepackage{amstext}
\usepackage{hyperref}
\tolerance=1000
\usepackage{amsmath}
\usepackage[bibstyle=alphabetic,citestyle=alphabetic,backend=bibtex]{biblatex}
\usepackage[small,compact]{titlesec}
\setlength{\textwidth}{19.7 cm}
\setlength{\hoffset}{-3.5 cm}
\setlength{\voffset}{-3.3 cm}
\addtolength{\textheight}{6.0 cm}

\setlength{\baselineskip}{12.1 pt}
\setlength{\parskip}{4.5 pt}
\setlength{\parindent}{0 pt}
\setlength{\abovedisplayskip}{0 pt}
\setlength{\belowdisplayskip}{0pt}
\setlength{\abovedisplayshortskip}{0 pt}
\setlength{\belowdisplayshortskip}{0 pt}

\bibliography{refs}
\AtEveryBibitem{\clearfield{doi}}
\AtEveryBibitem{\clearfield{url}}
\AtEveryBibitem{\clearfield{issn}}
\renewcommand*{\bibfont}{\footnotesize}

\author{Paul Bryan}
\date{}
\title{}
\hypersetup{
 pdfauthor={Paul Bryan},
 pdftitle={},
 pdfkeywords={},
 pdfsubject={},
 pdfcreator={Emacs 25.2.2 (Org mode 9.1.6)}, 
 pdflang={English}}
\begin{document}


\section*{D7 Rope - Bryan}
\label{sec:orga46faa9}

\subsection*{AMOUNT OF TIME AS AN ACTIVE RESEARCHER}
\label{sec:org71b965b}

It is 5.5 years since I graduated with my highest educational qualification. From 2012 to the present I have been employed at 1.0 FTE without interruption. 

\subsection*{RESEARCH OPPORTUNITIES}
\label{sec:orgb57b1c4}

From 2012-2015 I was an SEW Visiting Assistant Professor at UCSD. My role was 0.6 teaching, 0.4 research. For 2016-2017, I held Research Fellow positions at the ANU, Warwick University and the University of Queensland as well as a Riemman Fellow at Leibniz University. I had small teaching load of around 0.1 teaching, with the remaining 0.9 devoted to research. Currently I am funded by a DECRA at Macuqrie University for 2018-2020 with 0.8 time devoted to research under the DECRA and the remaining 0.2 to be distributed amongst teaching, administration and other projects such as the current proposal. I also hold a continuing Lecturer position at Macquarie which will become active in 2021 upon completion of the DECRA.

I received research mentoring from Prof. Ben Andrews throughout my PhD studies, and in the years after. I also had access to mentoring by leaders in the field, including Prof.'s Ben Chow, Lei Ni, Peter Topping and Joseph Grotowski over the past five years.

All of my appointments have provided access to library facilities, computing as well as opportunities for travel to conferences and other research institutions. I was awarded an AMS/Simons travel grant in 2014 that provided funding for such travel.

\subsection*{RESEARCH ACHIEVEMENTS AND CONTRIBUTIONS}
\label{sec:org5d9c0ea}

\subsubsection*{Awards}
\label{sec:org0446f7c}

\begin{itemize}
\item 2018--2020 Discovery Early Career Research Award
\item 2015 Riemann Fellow, Leibniz University
\item 2014 AMS-Simons Travel Award
\item 2008--2011 Australian Postgraduate Award
\item 2008--2011 Research School of Physical Sciences and Engineering Top Up
\item 2007--2008 ANU Honours Scholarship
\item 11/2006--03/2007 ANU Summer Research Scholarship
\end{itemize}

\subsubsection*{Invited speaker}
\label{sec:orgb62fc75}

\begin{itemize}
\item Dec 2017 Curve Shortening of Networks
61st AustMS - Geometric Analysis session
\item May 2017 Harnack Inequalities and Ancient Solutions of Hypersurface Flows
China-Australia joint conference on partial differential equations
\item Dec 2016 Harnack Inequalities and Ancient Solutions of Hypersurface Flows
Singularities of Geometric Partial Differential Equations, Warwick University
\item June 2016 Ancient soutions and Harnack Inequalities for curvature flows
Pontifical Catholic University of Chile
\item May 2016 Ancient solutions of curvature flows in the sphere
University of Warwick
\item November 2015 Harnack inequalities, Aleksandrov reflection, and ancient solution of curvature flows on the sphere.
Free University, Berlin
\item October 2015 Harnack inequalities and Classification of Ancient Solutions to Mean Curvature Flow
Riemann Center, Liebniz University Oberseminar
\item September 2015 Isoperimetric Inequalities: What's new since Queen Dido? An case for the isoperimetric profile
ANU Colloquium
\item September 2015 Harnack inequalities and Classification of Ancient Solutions to Mean Curvature Flow
Monash University Geometry and Analysis Seminar
\item August 2015 Harnack inequalities and Classification of Ancient Solutions to Mean Curvature Flow
ANU Geometry and Analysis Seminar
\item June 2015 Comparison results for the Isoperimetric profile and curvature flows
Columbia University Geometry Seminar
\item April 2015 A viscosity equation and applications of the maximum principle for the isoperimetric profile
Geometric Flows: Recent Developments and Applications (BIRS)
\item May 2014 Curvature Flows and Isoperimetry
UC Santa Cruz Colloquium
\item March 2014 Isoperimetric Problems in Geometry
UCI Geometry Seminar
\item October 2013 Polar Dual of Convex Bodies and Curvature Flows
UCSD Differential Geometry Seminar
\item May 2013 Isoperimetry and Viscosity Solutions in Geometric Evolution Equations
Pacific Northwest Geometry Seminar
\item January 2013 Brendle's proof of the Lawson conjecture
UCSD Differential Geometry Seminar
\item October 2012 Isoperimetric comparison techniques for Ricci flow on surfaces
UCSD Differential Geometry Seminar
\item Jun 2012 A Viscosity Equation for the Isoperimetric Profile of Surfaces
Peking University, Analysis Seminar
\item Sep 2011 An Isoperimetric Comparison Theorem for the Ricci Flow on Surfaces
55th Annual Australian Mathematical Society Meeting
\item Aug 2011 An Isoperimetric Comparison Theorem for the Ricci Flow on Surfaces
Peking University, Analysis Seminar
\item May 2011  An Isoperimetric Comparison Theorem for the Ricci Flow on Surfaces
ANU, Analysis Seminar
\item Mar 2009  An Isoperimetric Comparison Theorem for the Ricci Flow on the \$2\$-sphere
Workshop on Nonlinear Analysis, ANU
\item Oct 2008 A Distance Comparison Theorem for the Curve Shortening Flow
Workshop on Geometric Analysis, University of Wollongong
\item Feb 2007 The Curve Shortening Flow: Grayson's Theorem\\
AMSI/CSIRO The Big Day In
\end{itemize}

\subsubsection*{Research Support Income}
\label{sec:org93b0ef4}

\begin{itemize}
\item DE180100110: 2018-2020
\item 2015 Riemann Fellow, Leibniz University
\item 2014 AMS-Simons Travel Award
\item 2008--2011 Australian Postgraduate Award
\item 2008--2011 Research School of Physical Sciences and Engineering Top Up
\item 2007--2008 ANU Honours Scholarship
\item 11/2006--03/2007 ANU Summer Research Scholarship
\end{itemize}

\subsubsection*{Other Professional Activities}
\label{sec:orge3c6b98}

\begin{itemize}
\item Refereed 10 articles for Calc. Var. PDE, Crelle's Journal, Proc. AMS and others.
\item Reviewed 18 articles on MathSciNet Mathematical Reviews.
\item Joint organiser for Matrix workshop: Elliptic Partial Differential Equations Of Second Order: Celebrating 40 Years Of Gilbarg And Trudinger’s Book, 2017
\item Joint organiser for PDE session of 61st AustMS Meeting 2017
\item Guest Editor for Matrix Annals 2017
\item Supervised three undergraduate research projects (UCSD 2014, UQ 2017)
\item Master's Panel: Marielle One, UQ 2017
\item Honour's Panel: Zi Ou, UQ 2017
\item Assisted supervision of Marielle Ong's Master's Thesis under Masoud Kamgarpour, UQ 2017
\item Assisted supervision of Janelle Louise' Ph.D. Thesis under Bennet Chow, UCSD 2014
\end{itemize}

\subsubsection*{Research Impact}
\label{sec:orgd9998fb}

\begin{itemize}
\item With Ben Andrews and in my PhD thesis I developed groundbreaking new techniques for deducing curvature control in geometric evolution equations using isoperimetric estimates. We used these to give the simplest proofs yet available for the fundamental convergence theorems for Ricci flow on the two-sphere \cite{MR2729306} and for curve shortening flow in the plane \cite{MR2794630,MR2843240}. I further extended the techniques to the Ricci flow on arbitrary closed surfaces \cite{Bryan}, providing a unifying approach to the Ricci flow on surfaces. This work has been referenced by others in lecture notes and theses as the most accessible approach to the convergence results. The second part of the current proposal on controlling extremal profiles as viscosity subsolutions seeks to extend these techniques further.

\item With Mohammad Ivaki, Julian Scheuer and Janelle Louie, I investigated Harnack inequalities are their applications to ancient solutions of curvature \cite{2017arXiv170307493B,bryanlouie,2016arXiv160401694B,2015arXiv150802821B,2015arXiv151203374B}. We provided the first examples of Harnack inequalities for hypersurface flows in non-Euclidean backgrounds and further obtained such inequalities for the intrinsic, fully non-linear cross curvature flow (subject to an integrability condition). Given that obtaining a Harnack inequality for the Ricci flow was a tremendous achievement involving extremely delicate analysis, coupled with the fact that the Ricci flow is not fully non-linear, the Harnack for the cross curvature flow should be seems Our work classifying ancient solutions of hypersurface flows in the sphere obtains extremely broad results applying to any parabolic flow. In this area, and more broadly in the study of fully non-linear PDE, typical assumptions are homogeneity and/or convexity/concavity of the flow speed. Our results require no such assumption, making use of a powerful and geometric parabolic Aleksandrov reflection technique.

\item With Lashi Bandara \cite{2017arXiv171209287B}, we have shown how to obtain the existence and regularity of heat kernels on smooth manifolds with geometric singularities (so called "rough metrics"). This work has just begun, but already is attracting attention among harmonic analysts as many researchers have been working hard on constructing heat kernels in lower regularity situations. Our new innovations open a whole new avenue of exploring this cutting edge field.
\end{itemize}
\end{document}
